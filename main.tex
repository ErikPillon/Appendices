\documentclass{book}

\usepackage[english]{babel}
\usepackage[T1]{fontenc} 
\usepackage[utf8]{inputenc}

\usepackage{amsmath,amssymb}
\usepackage{amsfonts}
\usepackage{mathtools}
\usepackage{amsthm}

%% THEOREM STYLE
\theoremstyle{plain}
\newtheorem{thm}{Theorem}[section]
\newtheorem{lem}[thm]{Lemma}
\newtheorem{prop}[thm]{Proposition}
\newtheorem*{cor}{Corollary}

\theoremstyle{definition}
\newtheorem{mydef}{Definition}[section]
\newtheorem{conj}{Conjecture}[section]
\newtheorem{exmp}{Example}[section]

\theoremstyle{remark}
\newtheorem{rmk}{Remark}
\newtheorem*{rec}{Recall}
\newtheorem{note}{Note}

%% AMBIENTI MATEMATICI
\newcommand{\R}{\mathbb{R}}
\newcommand{\Rn}{\mathbb{R^n}}
\newcommand{\C}{\mathbb{C}}
\newcommand{\tu}{\tilde{U}}
\renewcommand{\hat}{\widehat}
\renewcommand{\phi}{\varphi}
\newcommand{\diff}{\mathop{}\!d}
\newcommand{\norm}[1]{\left\lvert #1\right\rvert}
\newcommand{\pd}[2]{\frac{\partial #1}{\partial #2}}
\newcommand{\pds}[2]{\frac{\partial^2 #1}{\partial #2^2}}

\usepackage{geometry}
\geometry{a4paper,top=3cm,bottom=3cm,left=3.5cm,right=3.5cm,	
	heightrounded,bindingoffset=5mm}

% --------- OTHER BEAUTIFUL FONTS --------------------------
% \usepackage{palatino}
% \usepackage{fontawesome}
% \usepackage{fourier}
% ----------------------------------------------------------
\usepackage{newtxtext}
\usepackage{newtxmath}

\usepackage{hyperref}
\frenchspacing 


\begin{document}

% !TeX encoding = UTF-8
% !TeX spellcheck = en_US
\section{Laplace transform}
\begin{flushright}
	\emph{" Je n'avais pas besoin de cette hypothèse-là."}\\ 
	
	Pierre-Simon, marquis de Laplace.
\end{flushright}
\label{sec:Laplace_transform}
The Laplace transform  $ \mathcal{L} $  is defined by
\[\mathcal{L}[f(t)](s)=\int_0^{\infty} f(t)e^{-st}\diff t\]
where $ f(t) $ is defined for $ t\geq0 $. 

The Laplace transform existence theorem states that, if $ f(t) $ is piecewise continuous on every finite interval in $ [0,\infty) $ satisfying $ \norm{f(t)}\leq Me^{at}$	  
for all $ t\in [0,\infty) $, then $ \mathcal{L}[f(t)](s) $ exists for all $ s>a $. 

The Laplace transform is also unique, in the sense that, given two functions $ F_1(t) $ and $ F_2(t) $ with the same transform so that
\[\mathcal{L}[F_1(t)](s)=\mathcal{L}[F_2(t)](s)=f(s),\] 	

then Lerch's theorem guarantees that the integral
$ \int_0^a N(t)\diff t=0 $ 	
vanishes for all $ a>0 $ for a null function defined by
$ N(t)=F_1(t)-F_2(t) $. 

The Laplace transform is linear since
\begin{align*}
	\mathcal{L}[af(t)+bg(t)]&=\int_0^{\infty}[af(t)+bg(t)]e^{-st}\diff t\\
	&=a\int_0^{\infty}fe^{-st}\diff t+b\int_0^{\infty}ge^{-st}\diff t\\		
	& =	a\mathcal{L}[f(t)]+b\mathcal{L}[g(t)].	
\end{align*}
The Laplace transform of a convolution is given by
\begin{align*}
	\mathcal{L}[f(t)*g(t)]=\mathcal{L}[f(t)]\mathcal{L}[g(t)]\\ \mathcal{L}^{-1}[FG]=\mathcal{L}^{-1}[F]*\mathcal{L}^{-1}[G].
\end{align*}
Now consider differentiation. Let $ f(t) $ be continuously differentiable $ n-1 $ times in $ [0,\infty) $. If $ \norm{f(t)}\leq Me^{at} $, then
\[
\mathcal{L}[f^{n}(t)](s)=s^n\mathcal{L}[f(t)]-s^{n-1}f(0)-s^{n-2}f'(0)-\ldots-f^{n-1}(0).\]
This can be proved by integration by parts,
\begin{align*}
	\mathcal{L}[f'(t)](s)&=\lim_{a\to\infty}\int_0^a e^{-st}f'(t)\diff t\\	
	&=\lim_{a\to \infty}\{\left[e^{-st}f(t)\right]_0^a +s\int_0^a e^{-st}f(t)\diff t\}	\\
	&=\lim_{a\to\infty}\left[e^{-sa}f(a)-f(0)+s\int_0^a e^{-st}f(t)\diff t\right]\\	
	&=\mathcal{L}[f(t)]-f(0).
\end{align*}
Continuing for higher-order derivatives then gives
\[\mathcal{L}[f''(t)](s)=s^2\mathcal{L}[f(t)](s)-sf(0)-f'(0). 	
\]
This property can be used to transform differential equations into algebraic equations, a procedure known as the \emph{Heaviside calculus}, which can then be inverse transformed to obtain the solution. For example, applying the Laplace transform to the equation
\[f''(t)+a_1f(t)+a_0f(t)=0\] 	
gives
\begin{align*}
	s^2\mathcal{L}[f(t)](s)-sf(0)-f'(0)+a_1\{s\mathcal{L}[f(t)](s)-f(0)\} +a_0\mathcal{L}[f(t)](s)=0\\
	\mathcal{L}[f(t)](s)(s^2+a_1s+a_0)-sf(0)-f'(0)-a_1f(0)=0, 	
\end{align*}
which can be rearranged to
\[\mathcal{L}[f(t)](s)=\frac{sf(0)+f'(0)+a_1f(0)}{s^2+a_1s+a_0}. 	\]
If this equation can be inverse Laplace transformed, then the original differential equation is solved.

The Laplace transform satisfies a number of useful properties. Consider exponentiation. If $ \mathcal{L}[f(t)](s)=F(s) $ for $ s>\alpha $ (i.e., $ F(s) $ is the Laplace transform of $ f $), then $ \mathcal{L}[e^{at}f](s)=F(s-a) $ for $ s>a+\alpha $. This follows from
\begin{align}
	F(s-a)\int_0^{\infty}fe^{-(s-a)t}\diff t=\int_0^{\infty}[f(t)e^{at}]e^{-st}\diff t	
	&=\mathcal{L}[e^{at}f(t)](s).
\end{align}		
The Laplace transform also has nice properties when applied to integrals of functions. If $ f(t) $ is piecewise continuous and $ \norm{f(t)}\leq Me^{at} $, then
\[\mathcal{L}\left[\int_0^tf(t')\diff t'\right]=\frac{1}{s}\mathcal{L}[f(t)](s).\] 	 


% !TeX encoding = UTF-8
% !TeX spellcheck = en_US
\section{Variable coefficients second order linear ODE}
\begin{flushright}
	\emph{"Je n'avais pas besoin de cette hypothèse-là."}\\ 
	
	Pierre-Simon, marquis de Laplace.
\end{flushright}
\label{sec:ODE_2nd}
Given functions $ p(x), q(x), f(x)\colon \R\to \R $, the differential equation in the unknown function $ y(x)\colon \R\to \R $ given by
\begin{equation}
y''(x)+p(x)y'(x)+q(x)y(x)=f(x)
\label{eq:diff_eq_2nd}
\end{equation}
is called a second order linear differential equation with variable coefficients. The equation in \eqref{eq:diff_eq_2nd} is called homogeneous iff for all $ x\in \R $ holds
\[f(x)=0. \]
The equation in \eqref{eq:diff_eq_2nd} is called of constant coefficients iff $ p(x), q(x), f(x) $ are constants.
\begin{thm}[Superposition property]
	If the functions $ y_1(x) $ and $ y_2(x) $ are solutions to the homogeneous linear equation
	\begin{equation}
	y''(x)+p(x)y'(x)+q(x)y(x)=0,
	\end{equation}
	then the linear combination $ c_1y_1(x)+c_2y_2(x) $ is also a solution for any constants $ c_1,c_2\in \R $.
\end{thm}
\begin{proof}
	Trivial.
\end{proof}
\subsubsection{Existence and uniqueness of solutions.}
\bigskip
\begin{thm}[Variable coefficients]
	If the functions $ p(x),q(x)\colon\left(x_1,x_2\right)\to \R $ are continuous, the constants $ x_0\in \left(x_1,x_2 \right)\to \R $ to the initial value problem
	\begin{equation}
	\begin{cases}
	y''(x)+p(x)y'(x)+q(x)y(x)=f(x),\\
	y(x_0)=y_0,\\
	y'(x_0)=y_1.
	\end{cases}
	\end{equation}
\end{thm}
\begin{rmk}
	Unlike the first order linear ODE where we have an explicit expression for the solution, there is no explicit expression for the solution of second order linear ODE.
\end{rmk}
\begin{rmk}
	Two integrations must be done to find solutions to second order linear. Therefore, initial value problems with two initial conditions can have a unique solution.
\end{rmk}
\begin{rec}
	Every solution of the first order linear equation
	\[y'(x)+p(x)y(x)=0 \]
	is given by $ y(x)=c e^{-A(x)}, $ with $ A(x)=\int_{0}^{x} p(\xi)\diff\xi. $
\end{rec}
\begin{rmk}
	The above statement is not true for solutions of second order, linear, homogeneous equations, $ y''(x)+p(x)y'(x)+q(x)y(x)=0. $
\end{rmk}
\subsubsection{The Wronskian}
\medskip
\begin{rmk}
	The Wronskian is a function that determines whether two functions are linear dependent or linearly independent.
\end{rmk}
\begin{mydef}
	The wronskian of functions $ y_1(x), y_2(x)\colon\left(x_1,x_2\right)\to \R $ is the function
	\[W_{y_1,y_2}(x)=y_1(x)y_2'(x)-y_1'(x)y_2(x). \]
\end{mydef}
Notice that if we denote by \[w(x)=\begin{bmatrix}
y_1(x) & y_2(x)\\
y_1'(x) & y_2'(x)
\end{bmatrix} \]
then we have that $ W_{y_1,y_2}(x)=\det(w(x)) $. 

If the functions $ y_i(x) $ are linearly dependent, then so are the columns of the Wronskian as differentiation is a linear operation, so the Wronskian vanishes. Thus, the Wronskian can be used to show that a set of differentiable functions is linearly independent on an interval by showing that it does not vanish identically. It may, however, vanish at isolated points. 
The following Theorem figure out what we've just said.
\begin{thm}[Wronskian]
	The continuously differentiable functions $ y_1,y_2\colon (x_1,x_2)\to \R $ are linearly dependent iff $ W_{y_1,y_2}(x)=0 $ for all $ x\in (x_1,x_2). $
\end{thm}
\subsection{General and fundamental solutions}
\begin{thm}
	If $ p(x), q(x)\colon (x_1,x_2)\to \R $ are continuous, then the functions $ y_1(x),y_2(x)\colon (x_1,x_2)\to\R $ solutions of the initial value problems 
	\begin{align*}
		y_1''(x)+p(x)y_1'(x)+q(x)y_1(x)=0, & y_1(0)=1, \qquad y_1'(0)=0,\\
		y_2''(x)+p(x)y_2'(x)+q(x)y_2(x)=0, & y_2(0)=0, \qquad y_2'(0)=1, 
	\end{align*}
	are linearly independent.
\end{thm}
\begin{rmk}
	Every linear combination $ y(x)=c_1y_1(x)+c_2y_2(x), $ is also a solution of the differential equation
	\begin{equation}
	\label{eq:def_homo}
	y''(x)+p(x)y'(x)+q(x)y(x)=0.
	\end{equation}
	Conversely, every solution $ y(x) $ of the equation above can be written as linear combination of the solutions $ y_1,y_2. $
\end{rmk}
The results above justify the following 
\begin{mydef}[Fundamental solution]
	Two solutions $ y_1,y_2 $ of the homogeneous equation are called fundamental solutions iff the functions $ y_1,y_2 $ are linearly independent, that is, iff $ W_{y_1,y_2}\neq 0. $
\end{mydef}
\begin{mydef}
	Given any two fundamental solutions $ y_1,y_2, $ and arbitrary constants $ c_1,c_2 $ the function
	\[y(x)=c_1y_1(x)+c_2y_2(x) \]
	is called the general solution of Eq.~\eqref{eq:def_homo}. 
\end{mydef}

\centering
{\Huge MISSING ABEL THEOREM}
% main reference:
% http://users.math.msu.edu/users/gnagy/teaching/13-spring/mth235/L07-235.pdf


\end{document}
